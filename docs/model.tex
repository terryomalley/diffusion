\documentclass[12pt]{article}

\usepackage{amsmath, amssymb, graphicx, hyperref}
\usepackage{booktabs}

\title{Modeling Sales Diffusion: Bass and Logistic Growth Models}
\author{}
\date{}

\begin{document}

\maketitle

\section{Introduction}
Sales diffusion models describe how new products are adopted over time. Two widely used approaches are:
\begin{itemize}
    \item The \textbf{Bass Model}, which captures diffusion via innovation and imitation effects.
    \item The \textbf{Logistic Growth Model}, which assumes an S-shaped growth curve driven by intrinsic market dynamics.
\end{itemize}
This document outlines the implementation, fitting, and evaluation of these models, emphasizing simplicity and robustness.

\section{The Bass Model}
The Bass Model explains cumulative adoption using two parameters: the \textit{innovation rate} (\(p\)) and the \textit{imitation rate} (\(q\)):
\[
S(t) = \left( p + q \frac{C(t)}{M} \right) \left( M - C(t) \right),
\]
where:
\begin{itemize}
    \item \(S(t)\): Sales at time \(t\),
    \item \(C(t)\): Cumulative sales up to time \(t\),
    \item \(M\): Market potential,
    \item \(p\): Probability of initial adoption,
    \item \(q\): Influence from prior adopters.
\end{itemize}
The model's S-shaped curve arises from the interaction between \(p\) and \(q\).

\subsection{Parameter Estimation}
The Bass Model parameters (\(p\), \(q\), and \(M\)) are estimated using Maximum Likelihood Estimation (MLE). Given observed sales data, the negative log-likelihood is:
\[
\text{NLL} = -\sum_{t=1}^T \log \mathcal{N}(S(t); \hat{S}(t), \sigma^2),
\]
where \(\hat{S}(t)\) are the model's predicted sales.

\section{The Logistic Growth Model}
The Logistic Growth Model describes adoption as a logistic function:
\[
S(t) = \frac{M}{1 + \exp(-r (t - t_{\text{inf}}))},
\]
where:
\begin{itemize}
    \item \(S(t)\): Sales at time \(t\),
    \item \(M\): Market potential (saturation level),
    \item \(r\): Growth rate,
    \item \(t_{\text{inf}}\): Time of inflection (maximum growth).
\end{itemize}
The model assumes symmetric growth around \(t_{\text{inf}}\).

\subsection{Parameter Estimation}
Similar to the Bass Model, parameters (\(M\), \(r\), \(t_{\text{inf}}\)) are estimated via MLE:
\[
\text{NLL} = -\sum_{t=1}^T \log \mathcal{N}(S(t); \hat{S}(t), \sigma^2).
\]

\section{Comparison of Models}
\subsection{Simulation}
Both models can simulate sales under varying parameter assumptions. For example:
\begin{itemize}
    \item \textbf{Bass Model}: Simulates the effect of high imitation (\(q\)) or high innovation (\(p\)).
    \item \textbf{Logistic Model}: Simulates different growth rates (\(r\)) or saturation levels (\(M\)).
\end{itemize}

\subsection{Fit Evaluation}
The fit of each model is assessed using:
\begin{itemize}
    \item \textbf{Root Mean Squared Error (RMSE)}:
    \[
    \text{RMSE} = \sqrt{\frac{1}{T} \sum_{t=1}^T (S_{\text{actual}}(t) - S_{\text{predicted}}(t))^2},
    \]
    \item \textbf{Mean Absolute Error (MAE)}:
    \[
    \text{MAE} = \frac{1}{T} \sum_{t=1}^T \left| S_{\text{actual}}(t) - S_{\text{predicted}}(t) \right|.
    \]
\end{itemize}

\section{Implementation}
\subsection{Bass Model}
The Bass Model was implemented as an R6 class with methods for simulation and parameter estimation. Simulation computes sales recursively:
\[
S(t) = \left( p + q \frac{C(t)}{M} \right) \left( M - C(t) \right),
\]
where cumulative sales are updated iteratively.

\subsection{Logistic Growth Model}
The Logistic Growth Model class computes sales directly from the logistic function. Parameter estimation uses numerical optimization to minimize the negative log-likelihood.

\section{Results}
The models were tested using simulated data:
\begin{itemize}
    \item \textbf{Simulation 1}: Data generated from the Bass Model, with parameters \(p = 0.03\), \(q = 0.4\), and \(M = 5000\).
    \item \textbf{Simulation 2}: Data generated from the Logistic Growth Model, with parameters \(r = 0.3\), \(t_{\text{inf}} = 6\), and \(M = 5000\).
\end{itemize}
Both models were fitted to each dataset, and fit metrics (RMSE, MAE) were computed.

\section{Conclusion}
The Bass and Logistic Growth models provide complementary approaches to sales diffusion modeling:
\begin{itemize}
    \item The Bass Model explicitly accounts for innovation and imitation effects.
    \item The Logistic Growth Model is simpler and assumes intrinsic growth dynamics.
\end{itemize}
Future work may include integrating covariates or exploring hybrid models.

\end{document}
